
% Alex, Paul, Shiny -- Sec. 4.3 -- Grouping of Events

\documentclass[12pt]{article}

\usepackage[backend=biber,
            citestyle=authoryear,
            bibstyle=authoryear,
            natbib]{biblatex}
\bibliography{L2_grouping.bib}

\begin{document}

\setcounter{section}{4}
\setcounter{subsection}{2}

\subsection{Grouping of Events}
In a many-to-one connection, detection reports for all 1024 beams are buffered.
These reports arrive at a chunked cadence set by the dedispersion process.
Once all beams have reported for a given chunk, candidate events (L1 headers)
are grouped in time, DM, and sky position, yielding a singular description for
bright bursts detected in multiple beams. The clustering is performed with a
simplified implementation of the DBSCAN algorithm~\citep{dbscan}, which has
$\mathcal{O}(n\log n)$ complexity and is capable of handling large event rates
without issue. After clustering, an L2 header is formed. This includes one or
more L1 headers, along with additional fields which are filled out downstream.
These fields include a refined position, improved RFI rating, known source
association, DM to maximum Galactic DM ratio, source class, and flux
estimation.
 
Position refinement is the first step after the L2 header is
formed. It is performed by comparing the detected S/Ns from each beam 
to what is expected from a frequency-dependent beam model. 
For multi-beam events, a lookup table 
containing relative S/N ratios for a grid of sky positions and spectral indices
is precomputed using the beam model. The detected S/Ns are then compared
to the lookup table to form a best guess of position and spectral index, 
given the detection pattern and recovered S/N values of the constituent beams. 
For single-beam events, a precomputed mapping between event S/N
and the location and uncertainty region is used. The mapping is constructed
from the distribution of simulated events drawn from the beam model that 
produce non-detections in adjacent beams. With a refined position and
estimation of what the on-axis S/N would be, the flux is estimated using the
radiometer equation applied to single pulses \citep{cordes_mclaughlin03}.

\printbibliography{}

\end{document}
